%#! cluttex --engine=platex --max-iterations=4 --makeindex='mendex -c -g -l -r -s ../sty/myjpbase.ist' --output-directory=out
\documentclass[a4paper,10pt,landscape,dvipdfmx]{jlreq}
\usepackage[T1]{fontenc}
\usepackage{txfonts}
\usepackage{okumacro}
\usepackage{pxrubrica}
\usepackage[german,japanese]{babel}

\usepackage{xcolor}
\usepackage{tocloft}
\usepackage{needspace}
\usepackage{tabularray}
\usepackage{fancyhdr}

\usepackage{sty/myMacros}%%%%%%%%% section / subsection /
                         %%%%%%%%% myTBLR環境定義
\usepackage{sty/myMacrosMakeIdx} % makeidx 周りのカスタマイズ

%% 作品索引のため hyper-link bookmark 機能を利用するとたくさんの警告メッセージ
%% が表示される。原稿のチェック時は、以下の sty/myMacrosHyperLink の取
%% り込みを行わないようコメントアウトすること
%%\usepackage{sty/myMacrosHyperLink} % mmakeidx hyper link 周りのカスタマイズ
%%\usepackage{bookmark}

\title{\uppercase{%%--COMPOSER--%%} \footnote{%%--COMPOSERINFO--%%}
  \\ Liste der Lieder
  \\ ~ %
  \\ ~ %
  \\ %%--作曲家--%% 歌曲集
}
\author{}

\begin{document}
\setlength{\headheight}{17.0pt}%
\addtolength{\topmargin}{-7.0pt}
\parindent = 0pt
\pagestyle{empty}
\selectlanguage{german}

\maketitle

\newpage
\pagestyle{fancy}
\fancyhf{} % clear all fields
\fancyfoot[L]{Inhaltsverzeichnis}%
\fancyfoot[C]{--- ~ \thepage ~ ---} %
\fancyfoot[R]{目次}%

\renewcommand{\contentsname}
{\makebox[.465\textwidth][l]{Inhaltsverzeichnis} ~ 目次}
\tableofcontents

\newpage
%%--INPUT--%%

\newpage
\phantomsection
\newcommand{\myIndexnameA}{Verzeichnis der Titel der Werke}
\newcommand{\myIndexnameB}{作品タイトル名索引}
\renewcommand{\indexname}{\myIndexnameA --- \myIndexnameB}

\pagestyle{fancy}
\fancyhf{} % clear all fields
\fancyfoot[L]{\myIndexnameA}%
\fancyfoot[C]{--- ~ \thepage ~ ---} %
\fancyfoot[R]{\myIndexnameB}%
\printindex

\end{document}
